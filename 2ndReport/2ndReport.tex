\documentclass{article}
\usepackage{ctex}
\usepackage{listings}
\usepackage{xcolor}

\lstset{
	inputencoding=utf8,
	extendedchars=\true,
	basicstyle=\ttfamily\small, % 基本样式:等宽字体,小号
	keywordstyle=\color{blue}\bfseries, % 关键字样式:蓝色,加粗
	commentstyle=\color{gray}, % 注释样式:灰色
	stringstyle=\color{red}, % 字符串样式:红色
	numbers=left, % 在左边显示行号
	numberstyle=\tiny\color{gray}, % 行号样式:灰色,小号字体
	frame=single, % 给代码块加框
	breaklines=true, % 代码过长时自动换行
	captionpos=b, % 标题位置在底部
	language=Bash, % 设定语言为 Bash
	escapechar=\%, % 使用 % 作为转义字符
	literate={␣}{\textvisiblespace}1 {汉}{\texttt{汉}}1 {字}{\texttt{字}}1, % 处理中文字符
}
\begin{document}
	
	\title{第二次实验课程报告}
	\author{谢泉飞23020007131}
	\date{\today} % 使用 \today 自动生成当前日期
	\maketitle
	
	\pagenumbering{roman}
	\tableofcontents
	\newpage
	
	\pagenumbering{arabic}
	
	\section{Introduction}
	linux基本操作,shell编程基本语法,正则表达式处理数据。
	\section{linux}
		\subsection{基本操作(5个)}
			\begin{lstlisting}[]
cd .. # 返回上一级目录
mkdir # 创建目录
touch # 创建文件
rm # 删除文件
mv # 移动或重命名文件
			\end{lstlisting}
			
		\subsection{常用操作(2)}
			\begin{lstlisting}[]
ls -l -t -h -a # 展示当前目录下文件,长形式、按时间排列、显示人类可理解文件大小、全部文件(包括隐藏)
chmod # 改变文件权限,执行文件前需要先给文件添加执行权限
			\end{lstlisting}
					
	\section{shell编程基本语法,vim编辑器的使用}
		\subsection{vim编辑器使用(3)}
			\begin{lstlisting}[]
vim newfile # 打开一个文件,如果没有则创建一个新文件后打开
# 在标准模式中使用HJKL进行左下上右的操作;
# 在标准模式按i进入插入模式;
# 在标准模式输入‘:’进入命令模式,之后输入wq保存并退出,需要对文件有写入权限才能修改和保存文件
			\end{lstlisting}
		\subsection{shell编程基本语法(5)}
			\begin{lstlisting}
echo # 打印
$ # 引用变量,可以理解为使用返回值
# 条件选择
cat condition.sh # 读取condition.sh文件内容
# 以下为其输出,同时展示条件分支的语法
			\end{lstlisting}
			\begin{lstlisting}
# 不知名报错,把上面代码行和此代码行放一起就会报错
if [ "$1"x = "bromikey"x ]; then
echo "Hello bromikey"
elif [ "$1"x = "dingzhen"x ]; then
echo "xuebao bizui"
else
echo "$1 have no access"
fi
			\end{lstlisting}
			\begin{lstlisting}
# 循环
cat loop.sh
#for (( i=0; i <= 100; i++ ))
#do
#sum=$[ $sum + $i ]
#done
#echo $sum
			\end{lstlisting}
			\begin{lstlisting}			
# 读取输入
cat read.sh
read -t 10 -p "enter your name" name
case "$name"x in
"dingzhen"x)
echo "xuebao bizui"
;;
"bromikey"x)
echo "hello $name"
;;
"158"x)
echo "zuiren"
;;
esac
			\end{lstlisting}
	\section{数据处理,正则表达式(5)}
		主要就是使用grep查找,sed替换这两个命令。
			\begin{lstlisting}	
grep 'error' logfile.txt				
			\end{lstlisting}
					
			\begin{lstlisting}	
grep '^a' filename.txt				
			\end{lstlisting}
				
					
			\begin{lstlisting}	
sed 's/foo/bar/g' input.txt > output.txt
			
			\end{lstlisting}
				
					
			\begin{lstlisting}	
ls | grep '\.log$'
			
			\end{lstlisting}
			\begin{lstlisting}	
grep -c '[0-9]' data.csv
			\end{lstlisting}
	\section{解题感悟}
	结果这些也没什么贴的,代码就是结果了,其他内容写在我对hexo静态博客上了。https://bromikey.github.io/
	
	仓库链接https://github.com/BroMikey/latex-learning
	
	对解题没什么感悟,毕竟是按照教程学的,有什么不会也直接问ai了,这些东西太多,肯定是记不完的,有个这个用法的印象,用的时候查就是了。目前我对正则的感受就是这些。
	
	不过对latex我感悟挺重的,这玩意是真不好用啊,还会报一些不知道为什么的错,比如texstudio\_OCVqlj.tex: 错误: 69: Undefined control sequence. then,这东西你说因为什么报错呢,then关键词?把这一行删了下一行开始报错,把上一行删了也还报错,不知名报错越用越难受,markdown明明更适合代码类的实验报告,还是不太理解学习这个工具的作用。	希望老师能给学生解答疑惑。
	
\end{document}
