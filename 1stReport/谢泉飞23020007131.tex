\documentclass{article}
\usepackage{ctex}
\usepackage{listings}
\usepackage{xcolor}

\lstset{
	inputencoding=utf8,
	extendedchars=\true,
	basicstyle=\ttfamily\small, % 基本样式:等宽字体,小号
	keywordstyle=\color{blue}\bfseries, % 关键字样式:蓝色,加粗
	commentstyle=\color{gray}, % 注释样式:灰色
	stringstyle=\color{red}, % 字符串样式:红色
	numbers=left, % 在左边显示行号
	numberstyle=\tiny\color{gray}, % 行号样式:灰色,小号字体
	frame=single, % 给代码块加框
	breaklines=true, % 代码过长时自动换行
	captionpos=b, % 标题位置在底部
	language=Bash, % 设定语言为 Bash
	escapechar=\%, % 使用 % 作为转义字符
	literate={␣}{\textvisiblespace}1 {汉}{\texttt{汉}}1 {字}{\texttt{字}}1, % 处理中文字符
}
\begin{document}
	
	\title{Your Document Title}
	\author{谢泉飞23020007131}
	\date{\today} % 使用 \today 自动生成当前日期
	\maketitle
	
	\pagenumbering{roman}
	\tableofcontents
	\newpage
	
	\pagenumbering{arabic}
	
	\section{Introduction}
	第一次使用latex编辑,不太习惯使用,其中git内容使用的markdown格式,发布在我的静态网站上https://bromikey.github.io/
	两个仓库链接是
	
	
	\section{Hello Git!}
	\subsection{简单的常用指令}
		\begin{lstlisting}
git status      # 查看当前文件状态
git branch      # 查看分支情况
git branch new-branch  # 创建新分支
git switch branch-name # 切换到该分支
git switch -c new-branch   # 创建并切换到该分支
git add filename   # 将文件提交到缓存区
git add -A     # 将所有更新的文件添加
git commit     # 将缓存区的文件提交到分支上
git commit -m""	# 同上不过直接指定message内容为""内内容
		\end{lstlisting}
	\subsection{在本地提交树上移动}
		\subsubsection{HEAD分离}
			HEAD一般是对当前分支的符号引用,指向你正在其基础上进行工作的提交记录。HEAD 总是指向当前分支上最近一次提交记录。大多数修改提交树的 Git 命令都是从改变 HEAD 的指向开始的。
			HEAD 通常情况下是指向分支名的(如 bugFix)。在你提交时,改变了 bugFix 的状态,这一变化通过 HEAD 变得可见。
			\begin{lstlisting}
git switch c1	# 将HEAD分离出来
			\end{lstlisting}
		\subsubsection{相对引用}
			我们直接使用指定哈希值来移动,例如一次提交值为`fed2....`,可以使用git switch fed2来使HEAD指向此提交
			不过,相对引用解决了这样指定的问题,可以使用`HEAD\^`来表示向上移动一个提交记录,`HEAD\~3`等来移动多个单位。
			这个操作符可以反复在节点名称后。
			
		\subsubsection{撤销变更}
			\begin{lstlisting}
git reset HEAD^
			\end{lstlisting}
			

	\subsection{本地与远程仓库的交互}
		\begin{lstlisting}
git fetch	# 获取远程仓库更新内容
git merge	# 将另一分支和当前分支合并
git pull	# 作用等同于git fetch + git merge
git push	# 将本地更新提交到远程仓库
git pull --rebase # 特定情况使用
		\end{lstlisting}
		
	\section{Hello Latex!}	
	\subsubsection{基本的内容}
	一下内容包含着一个文档基本内容。
		\textbackslash documentclass[a4paper, 12pt]\{article\}
		
		\textbackslash begin\{document\}\\
		\textbackslash title\{My First Document\}\\
		\textbackslash author\{My Name\}\\
		\textbackslash date\{\textbackslash today\}\\
		\textbackslash maketitle\\
		
		\textbackslash section\{Introduction\}\\
		This is the introduction.
		
		\textbackslash section\{Methods\}\\
		
		\textbackslash subsection\{Stage 1\}\\
		\textbackslash label\{sec1\} The first part of the methods.\\
		
		\textbackslash subsection\{Stage 2\}\\
		The second part of the methods.\\
		
		\textbackslash section\{Results\}\\
		Here are my results. Referring to section \textbackslash ref\{sec1\} on page \textbackslash pageref\{sec1\}\\
		\textbackslash end\{document\}\\
		
		
	\subsubsection{中文内容显示}
	此部分内容解决中文内容无法显示的问题。
		\textbackslash usepackage\{ctex\}\\
		
	\subsubsection{代码显示}
	此部分内容为了正确显示bash代码。
		\textbackslash lstset\{
		inputencoding=utf8,\\
		extendedchars=\textbackslash true,\\
		basicstyle=\textbackslash ttfamily\textbackslash small, \% 基本样式:等宽字体,小号\\
		keywordstyle=\textbackslash color\{blue\}\textbackslash bfseries, \% 关键字样式:蓝色,加粗\\
		commentstyle=\textbackslash color\{gray\}, \% 注释样式:灰色\\
		stringstyle=\textbackslash color\{red\}, \% 字符串样式:红色\\
		numbers=left, \% 在左边显示行号\\
		numberstyle=\textbackslash tiny\textbackslash color\{gray\}, \% 行号样式:灰色,小号字体\\
		frame=single, \% 给代码块加框\\
		breaklines=true, \% 代码过长时自动换行\\
		captionpos=b, \% 标题位置在底部\\
		language=Bash, \% 设定语言为 Bash\\
		escapechar=\textbackslash\%, \% 使用 \% 作为转义字符\\
		literate=\{␣\}\{\textbackslash textvisiblespace\}1 \{汉\}\{\textbackslash texttt\{汉\}\}1 \{字\}\{\textbackslash texttt\{字\}\}1, \% 处理中文字符\\
		\}\\
		
\end{document}
